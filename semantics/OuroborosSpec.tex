\documentclass{article}
\usepackage[margin=2.5cm]{geometry}
\usepackage{amsmath, amssymb, stmaryrd}
\usepackage{mathpazo, times}
\usepackage{float}
\usepackage{listings}
\usepackage{url}
\usepackage{parskip}


\newcommand{\sExtend}{\mathrel{\triangleright}}

\newcommand{\sPar}{\mathrel{\parallel}}
\newcommand{\sProc}[2]{{#1}_{#2}}

\newcommand{\sSystem}[2]{\left( #1, #2 \right)}
\newcommand{\sProcess}[4]{\left( U, sk, #1, #2, #3, #4 \right)}

\newcommand{\sChain}{C}
\newcommand{\sChains}{\mathbb{C}}
\newcommand{\sState}{st}
\newcommand{\sSlot}{sl}
\newcommand{\sLeader}{\mathsf{F}}

\newcommand{\sQueue}{\mathbb{Q}}
\newcommand{\sProcesses}{\mathbb{P}}
\newcommand{\sBlock}[4]{\left( #1 , #2 , #3 , #4 \right)}

\newcommand{\sMaxValid}{\mathsf{maxValid}}
\newcommand{\sValid}{\mathsf{valid}}
\newcommand{\sHash}{\mathsf{H}}
\newcommand{\sHead}{\mathsf{head}}
\newcommand{\sSign}[4]{\mathsf{Sign}_{#1}\left(#2, #3, #4\right)}


\floatstyle{boxed}
\restylefloat{figure}

\lstset{basicstyle=\ttfamily\small}

\begin{document}

\title{Ouroboros blockchain protocol specification (DRAFT)}
\author{Duncan Coutts\footnote{Well-Typed and IOHK, \texttt{duncan@well-typed.com}, \texttt{duncan.coutts@iohk.io}}}
\date{\today}

\maketitle

\section{Introduction}

This document is intended to be an alternative formulation of the Ouroboros
blockchain protocol \cite{ouroboros}. The purpose is to describe the protocol
executed by honest parties in a precise operational style, and to be concise
by omitting proofs and other details needed to support proofs (such as the
possible behaviour of adversaries), but to otherwise be faithful to the
original paper.

\section{Preliminaries}

A time \emph{slot}, ranged over by $\sSlot \in \mathsf{Slot}$ where
$\mathsf{Slot} \cong \mathbb{N}$, represents a
period of clock time of a fixed duration.

An \emph{epoch}, ranged over by $e_j \in \mathbb{N}$, identifies a sequence of
time slots $[jR \ldots j(R+1) - 1]$ where the parameter $R \in \mathbb{N}$
is the fixed epoch length.

TODO: block, chain, hash function including codomain of hash

A \emph{user} or \emph{party}, ranged over by $U \in \mathsf{User}$,
represents an identifier for a stakeholder process taking part in the
protocol. We sometimes write $U_i \in \{U_1, \ldots, U_n \}$ when describing
a party within the context of all the parties taking part in the protocol,
however user identifiers are not necessarily dense integers.

A \emph{process}, ranged over by $P \in \mathsf{Process}_\pi$ is a
state tuple. The exact type of this state tuple is different for each version
of the protocol and will be described below in the context of each protocol.
For example, in the first protocol $\pi_{\text{SPoS}}$, a process
$P$ is a tuple $\sProcess{\sChain}{\sState}{\rho}{\sSlot}$.

A \emph{process group}, ranged over by $\mathbb{P} \in \mathsf{ProcessGroup}$,
is either empty, a single process $P$ or a combination of process
groups written $\mathbb{P}_1 \sPar \mathbb{P}_2$. The typical notation is
$\sProcesses \sPar ( a, b ... )$ to denote a single process state
tuple in the context of a group of other processes.

A \emph{broadcast network}, ranged over by $\sQueue \in \mathsf{BroadcastNetwork}$, consists of a set of pairs $\mathsf{Chain} \times \mathsf{User}$.
Each pair represents a message addressed to a particular user.

The overall \emph{system} is a tuple $\mathsf{ProcessGroup} \times \mathsf{BroadcastNetwork}$

The rules below are expressed as either the initial state of the system or
as valid transitions on the state of the system, in the form

\begin{equation*}
\frac{
  \text{conditions and definitions}
}{
  \sSystem{ \sProcesses }{ \sQueue }
\rightarrow
  \sSystem{ \cdots }{ \cdots }
} \textsc{rule name}
\end{equation*}

\section{Protocol $\pi_{\text{SPoS}}$}

This is the simplest of the four versions of the protocol from the paper
\cite{ouroboros}.

In this protocol, a $P \in \mathsf{Process}_{\pi_{\text{SPoS}}}$ is
a state tuple $\sProcess{\sChain}{\sState}{\rho}{\sSlot} \in 
\mathsf{User} \times \mathsf{PrivKey} \times \mathsf{Chain} \times \mathsf{Hash} \times \mathsf{Aux} \times \mathsf{Slot}$

The protocol has a number of global parameters.
\begin{itemize}
\item The initial users $U_1, \ldots, U_n$ and their keys and stakes $(\mathsf{sk}_1, \mathsf{vk}_1, s_1), \ldots, (\mathsf{sk}_n, \mathsf{vk}_n, s_n)$
\item A leader selection function $\sLeader \in \mathsf{Aux} \times \mathsf{Slot} \rightarrow \mathsf{User}$
\item A PRNG seed $\rho \in \mathsf{Aux}$ used by the leader selection function
\item A transaction data function $\mathsf{T} \in \mathsf{Slot} \rightarrow \mathsf{Data}$.
\end{itemize}

Note that the transaction data function is not intended to be realistic,
it simply models the otherwise unspecified source of transaction data in
this version of the protocol description.

\bigskip

\begin{equation*}
\frac{
  \begin{array}{r@{\;}c@{\;}l@{\qquad}l@{\;}c@{\;}l}
    \mathbb{U} & = & \{U_1, \ldots, U_n\}
    &
    B_0 & = & \left( (\mathsf{vk}_1, s_1), \ldots, (\mathsf{vk}_n, s_n), \rho \right)
    \\[1mm]
    \sChain & = & B_0
    &
    \sState & = & \sHash(\sHead(\sChain)
  \end{array}
}{
  \sSystem{ \bigparallel_{U \in \mathbb{U}} \sProcess{\sChain}{\sState}{\rho}{0}}{\emptyset}
} \textsc{init}
\end{equation*}

\bigskip

\begin{equation*}
\frac{
\begin{array}{l@{\;}c@{\;}l@{\qquad}l@{\;}c@{\;}l}
  \sChains & = & \left\{ \sChain' | (\sChain', U') \in \sQueue, U' = U, \sValid(sl, \sChain') \right\}
  &
  \sChain' & = & \sMaxValid(\sChain, \sChains)
  \\
  \sQueue' & = & \left\{ \sChain' | (\sChain', U') \in \sQueue, U' \neq U \right\}
  &
  \sState' & = & \sHash(\sHead(\sChain'))
\end{array}
}{
  \sSystem{\sProcesses \sPar \sProcess{\sChain}{\sState}{\rho}{\sSlot} }{\sQueue}
\rightarrow
  \sSystem{\sProcesses \sPar \sProcess{\sChain'}{\sState'}{\rho}{\sSlot} }{\sQueue'}
} \textsc{receive}
\end{equation*}

\bigskip

\begin{equation*}
\frac{
    \sLeader(\rho, sl) \neq U
}{
  \sSystem{\sProcesses \sPar \sProcess{\sChain}{\sState}{\rho}{\sSlot}}{\sQueue}
\rightarrow
  \sSystem{\sProcesses \sPar \sProcess{\sChain}{\sState}{\rho}{\sSlot+1}}{\sQueue}
} \textsc{skip}
\end{equation*}

\bigskip

\begin{equation*}
\frac{
    \begin{array}{c}
    \sLeader(\rho, sl) = U
    \\[1mm]
    B = \sBlock{\sState}{d}{\sSlot}{\sigma}
    \quad
    \sigma = \sSign{sk}{\sState}{d}{\sSlot}
    \quad
    d = \mathsf{T}(\sSlot)
    \\[1.5mm]
    \sChain' = \sChain \sExtend B
    \qquad
    \sQueue' = \sQueue \cup \{ (\sChain', U) | U \in \mathbb{U} \}
    \end{array}
}{
  \sSystem{\sProcesses \sPar \sProcess{\sChain}{\sState}{\rho}{\sSlot}}{\sQueue}
\rightarrow
  \sSystem{\sProcesses \sPar \sProcess{\sChain}{\sState}{\rho}{\sSlot+1}}{\sQueue'}
} \textsc{diffuse}
\end{equation*}

\bigskip

Meta-rules not currently encoded:
\begin{itemize}
\item Order: each process $P \in \mathbb{P}$ must $\textsc{receive}$ then
either $\textsc{skip}$ or $\textsc{diffuse}$.
\item Progress of time: all processes in $\mathbb{P}$ must $\textsc{skip}$ or $\textsc{diffuse}$
once in each round.
\end{itemize}

Definitions currently missing
\begin{itemize}
\item block \& chain
\item hash function, signature function, verification predicate
\item valid block function, including:
      \begin{itemize}
      \item sig checking,
      \item chaining validity
      \item blocks must come from the past, ie slot $sl' < sl$
      \end{itemize}
\item the maxValid function
\item PRNG/Aux type
\end{itemize}


\section{Protocol $\pi_{\text{DPoS}}$}

TODO

\bibliographystyle{apalike}
\bibliography{references}

\end{document}
