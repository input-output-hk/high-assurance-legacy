\documentclass[a4paper,11pt]{article}

\usepackage{typearea}

\usepackage{lmodern}
\usepackage[T1]{fontenc}
\usepackage{textcomp}

\usepackage{isabelle,isabellesym}

\usepackage{latexsym}
\usepackage{amssymb}
\usepackage{eufrak}

\usepackage{pdfsetup}

\urlstyle{rm}
\isabellestyle{it}

\begin{document}

\title{The $\chi$-Calculus}
\author{Wolfgang Jeltsch\\\small\texttt{wolfgang@well-typed.com}}

\maketitle

\tableofcontents

\parindent 0pt\parskip 0.5ex

\section{Introduction}

The $\chi$-calculus is a process calculus with the following features:
\begin{itemize}

\item

The only supported notion of communication is asynchronous unicast communication. Other notions of
communication have to be emulated based on that. Asynchronism is ensured by disallowing sequential
composition of an output operation and some follow-up process.

\item

Processes can work with arbitrary data. The $\chi$-calculus uses higher-order abstract syntax (HOAS)
to make the full computational power of the host language available inside the process calculus. It
also uses HOAS for dealing with locally created channels. Through the consequent use of HOAS, all
naming issues are handled by the host language.

\item

Process terms can be infinite. This allows for recursively defined processes. Replication of
processes (provided by the $!$-operator in the $\pi$- and the $\psi$-calculus) is not provided by
the $\chi$-calculus, since the possibility of using recursion makes it unnecessary.

\end{itemize}

\input{session}

\end{document}
