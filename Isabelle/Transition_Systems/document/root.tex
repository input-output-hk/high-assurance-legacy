\documentclass[a4paper,11pt]{article}

\usepackage{typearea}

\usepackage{isabelle,isabellesym}

\usepackage{latexsym}

\usepackage{pdfsetup}

\urlstyle{rm}
\isabellestyle{it}

\usepackage{amsmath}

\newcommand{\openandsendexample}{
  (\nu y)P
  \overset{\overline{x}\hspace{-0.03em}(\hspace{-.06em}y\hspace{-.12em})}{\longrightarrow}
  Q
}

\begin{document}

\title{A Theory of Labeled Transition Systems\\with Support for Contexts and Scope Openings}
\author{Wolfgang Jeltsch\\\small\texttt{wolfgang@well-typed.com}}

\maketitle

\tableofcontents

\parindent 0pt\parskip 0.5ex

\section{Introduction}

We present a theory of labeled transition systems, formalized in Isabelle/HOL. Our theory supports
two advanced features:
\begin{description}

\item[Contexts.]

Whether a particular transition is possible may depend on a context. This feature is useful, for
example, for implementing process calculi with named processes. The general idea is that a context
maps names to processes and that a transition $\langle N\rangle \overset{\alpha}{\longrightarrow}
P$, where $\langle N\rangle$ denotes the process with name~$N$, is possible in a context~$\Gamma$ if
and only if $\Gamma(N) \overset{\alpha}{\longrightarrow} P$ is possible in~$\Gamma$.

\item[Scope openings.]

A transition may open scopes by having the label contain binders that bind their respective names
also in the target process. In the $\pi$-calculus, for example, a transition
$\openandsendexample$ is possible if $P \overset{\overline{x}y}{\longrightarrow} Q$ is. The latter
transition just sends $y$ along~$x$, but the former first opens the $y$-scope and only then sends
$y$ along~$x$. The channel denoted by~$y$ is accessible in~$Q$ because the $(y)$ in the label
$\overline{x}(y)$ binds $y$ in~$Q$. Certain other process calculi, like $\psi$-calculi and the
$\chi$-calculus, even allow for opening multiple scopes in a single transition. Our theory of
transition systems assumes that binding structures that realize scope openings are implemented using
higher-order abstract syntax (HOAS).

\end{description}

\input{session}

\end{document}
