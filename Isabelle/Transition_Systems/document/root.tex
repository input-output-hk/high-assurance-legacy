\documentclass[a4paper,11pt]{article}

\usepackage{typearea}

\usepackage{isabelle,isabellesym}

\usepackage{latexsym}

\usepackage{pdfsetup}

\urlstyle{rm}
\isabellestyle{it}

\usepackage{amsmath}

\begin{document}

\title{Transition Systems}
\author{Wolfgang Jeltsch}

\maketitle

\tableofcontents

\parindent 0pt\parskip 0.5ex

\section{Introduction}

We present a theory of labeled transition systems, formalized in Isabelle/HOL. Our theory supports
two advanced features:
\begin{description}

\item[Contexts.]

Whether a particular transition is possible may depend on a context. This feature can be used, for
example, to implement process calculi with named processes. In such a calculus, a context maps names
to processes, and a transition $\langle N\rangle \overset{\alpha}{\longrightarrow} P$, where
$\langle N\rangle$ denotes the process with name~$N$, is possible in a context~$\Gamma$ if and only
if $\Gamma(N) \overset{\alpha}{\longrightarrow} P$ is possible in~$\Gamma$.

\item[Scope openings.]

A transition may open scopes by having the label contain binders that bind their respective names
also in the target process. In the $\pi$-calculus, for example, a transition
$(\nu y)P \overset{\overline{x}\hspace{-0.03em}(\hspace{-.06em}y\hspace{-.12em})}{\longrightarrow}
Q$ (which opens the $y$-scope and sends $y$ along~$x$) is possible if $P
\overset{\overline{x}y}{\longrightarrow} Q$ (which sends $y$ along~$x$ without opening the
$y$-scope) is possible. In the former transition, $y$ is bound in~$Q$ via the $(y)$ in the label.
Certain other process calculi, like $\psi$-calculi and the $\chi$-calculus, allow for opening
multiple scopes in a single transition.

\end{description}

\input{session}

\end{document}
