\documentclass{article}
\usepackage[margin=2.5cm]{geometry}

\usepackage{kpfonts}
\usepackage{amsmath}
\usepackage{hyperref}
\usepackage{todonotes}

\newcommand{\pOut}{{!}}
\newcommand{\bOut}{{!\!!}}
\newcommand{\pInp}{{?}}
\newcommand{\bInp}{{?\!?}}
\newcommand{\new}[2]{(\nu\!#1) #2}
\newcommand{\freshFor}{\mathrel{\#}}
\newcommand{\rec}{\mathtt{rec}}
\DeclareMathOperator{\bn}{bn}
\DeclareMathOperator{\n}{n}

\begin{document}

\title{Simplified Semantics}
\author{Edsko de Vries}
\maketitle

\cite{Borgström2015}
stays close to ``Broadcast Psi-calculi with an Application to Wireless Protocols''

\cite{AmanPohjola:2016:BUT:2854065.2854080} up-to

\section*{Operational Semantics}

Let $P, Q$ range over processes; $\mathcal{F}$ range over equivariant functions
from values to processes; $c, a, b$ over channels; $V$ over values.

\begin{itemize}

\item Point-to-point communication

\begin{equation*}
%
% PIn
%
\frac{
}{
c \pInp \mathcal{F} \xrightarrow{c \pInp V} \mathcal{F}(V)
}
\mathtt{In}
%
\qquad
%
% POut
%
\frac{
}{
c \pOut V. P \xrightarrow{c \pOut V} P
}
\mathtt{Out}
%
\qquad
%
\frac{
P \xrightarrow{\new{\bar{a}}{c \pOut V}} P' \qquad
Q \xrightarrow{c \pInp V} Q' \qquad
\bar{a} \freshFor Q
}{
P \parallel Q \xrightarrow{\tau} \new{\bar{a}}{(P' \parallel Q')}
}
\mathtt{Com}
%
\end{equation*}
%
\begin{equation*}
\frac{
P \xrightarrow{\new{\bar{a}}{c \pOut V}} P' \qquad
b \freshFor \bar{a}, c \qquad
b \in \n(V)
}{
\new{b}{P} \xrightarrow{\new{\bar{a},b}{c \pOut V}} P'
}
\mathtt{Open}
\end{equation*}
%
where
%
\begin{itemize}
\item Symmetric version of $\mathtt{Com}$ elided.
\item Rule $\mathtt{Com}$ doubles as a closing rule.
\end{itemize}

\todo[inline]{The psi calculus paper states some additional assumptions here,
but I \emph{think} they online pertain to frames, which we make no use of. But
we may need an additional freshness requirement somewhere.}

\item Broadcast communication

\begin{equation*}
%
% BOut
%
\frac{
}{
\bOut V. P \xrightarrow{\bOut V} P
}
\mathtt{BrOut}
%
\qquad
%
% BrIn
%
\frac{
}{
\bInp \mathcal{F} \xrightarrow{\bInp V} \mathcal{F}(V)
}
\mathtt{BrIn}
%
\qquad
%
% BCom
%
\frac{
P \xrightarrow{\new{\bar{a}}{\bOut V}} P' \qquad
Q \xrightarrow{\bInp V} Q' \qquad
\bar{a} \freshFor Q
}{
P \parallel Q \xrightarrow{\new{\bar{a}}{\bOut V}} P' \parallel Q'
}
\mathtt{BrCom}
\end{equation*}
%
\begin{equation*}
%
% BrOpen
%
\frac{
P \xrightarrow{\new{\bar{a}}{\bOut V}} P' \qquad
b \freshFor \bar{a} \qquad
b \in \n(V)
}{
\new{b}{P} \xrightarrow{\new{\bar{a},b}{\bOut V}} P'
}
\mathtt{BrOpen}
%
\qquad
%
% Close
%
\frac{
P \xrightarrow{\new{\bar{a}}{\bOut V}} P'
}{
P \xrightarrow{\tau} \new{\bar{a}}{P'}
}
\mathtt{BrClose}
\end{equation*}
%
\begin{equation*}
%
% BrMerge
%
\frac{
P \xrightarrow{\bInp V} P' \qquad
Q \xrightarrow{\bInp V} Q'
}{
P \parallel Q \xrightarrow{\bInp V} P' \parallel Q'
}
\mathtt{BrMerge}
\end{equation*}
%
where
%
\begin{itemize}
\item Symmetric version of rule $\mathtt{BrCom}$ is elided; note that the psi
calculus paper also elides rule $\mathtt{BrOpen}$ (we stated it explicitly).
\item Rule $\mathtt{BrMerge}$ would not be necessary if we had an explicit rule
\begin{math}
\displaystyle
\frac{
P \equiv P' \xrightarrow{\alpha} Q' \equiv Q
}{
P \xrightarrow{\alpha} Q'
}
\end{math}
if we want such a rule to be \emph{derivable} instead (which makes the semantics
easier to use) rule $\mathtt{BrMerge}$ is needed.
\item Rule $\mathtt{BrClose}$ is slightly more permissive than its equivalent in
the psi calculus presentation, where it only applies at the point where the
channel is bound. I'm not entirely sure why that restriction is in place; it
is unusual (the closing rule for point-to-point communication has no similar
restriction).
\item The broadcast rules together basically means that when a process is ready
to broadcast, the semantics allow it to couple with any number (zero or more)
of input processes, the communication happens in one step, and then the output
is no longer available after that; hence, this already models unreliable
delivery.
\item Since we assume the existance of a single, global, broadcast channel,
restriction does not apply to broadcast channels. However, it \emph{is} possible
to extrude a point-to-point communication channel through a broadcast communication.
\end{itemize}

\item Congruence rules

\begin{equation*}
%
% Par
%
\frac{
P \xrightarrow{\alpha} P' \qquad
\bn(\alpha) \freshFor Q
}{
P \parallel Q \xrightarrow{\alpha} P' \parallel Q
}
\mathtt{Par}
%
\qquad
%
% Scope
%
\frac{
P \xrightarrow{\alpha} P' \qquad
a \freshFor \alpha
}{
\new{a}{P} \xrightarrow{\alpha} \new{a}{P'}
}
\mathtt{Scope}
%
\qquad
%
% Rec
%
\frac{
\mathcal{F}(\rec \; \mathcal{F}) \xrightarrow{\alpha} P'
}{
\rec \; \mathcal{F} \xrightarrow{\alpha} P'
}
\mathtt{Rec}
\end{equation*}

Symmetric version of $\mathtt{Par}$ elided.

\item{Network Limitations}

TODO

\end{itemize}

\section*{Differences from the Psi Calculus}

\begin{itemize}
\item Asynchronous? --- implies unordered
\item no assertions
\item single broadcast channel
\item single kind of broadcast -- explain how it can be encoded (refer to the paper)
\item explicitly allow for arbitrary functions after input; this is how we want to use it \emph{anyway}
\item no up-to techniques available for the broadcast extension
\end{itemize}

\section*{References}

pi-calculus in nominal logic: \url{https://arxiv.org/pdf/0809.3960.pdf}

\bibliographystyle{acm}
\bibliography{references}

\end{document}
